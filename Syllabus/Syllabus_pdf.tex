\documentclass[12pt]{article}
\usepackage[margin=1in]{geometry} 
\usepackage{xcolor}
\usepackage{hyperref}
\hypersetup{
	colorlinks=true,
	linkcolor=blue!50!red,
	urlcolor=blue!
}

\begin{document}

\begin{center}
\textbf{\Large{2020 Python Boot Camp Syllabus}}
\end{center}


\section*{Educational Aim}
The goal of our Python Boot Camp is to provide you with the skills needed to produce
a portfolio worthy data science/machine learning project.

\section{Topics to be Covered}
In fitting with our educational aim, this course will walk you through the steps
involved in a typical data science/machine learning project. While covering a variety
of important techniques and algorithms we will illuminate themes and motivations
that can be adapted to many data science settings.

Below we outline the specific topics we will be covering in this boot camp.
\begin{itemize}
	\item Data Gathering Techniques
	\begin{itemize}
		\item Searching common online sources for data
		\item Basic web scraping with BeautifulSoup
		\item Interacting with APIs
	\end{itemize}
	\item Data Cleaning
	\begin{itemize}
		\item Data Types
		\item Basic data exploration with pandas, and numpy
		\item Basic plotting with matplotlib
		\item Handling Missing Data
		\item Common Data Transformations
	\end{itemize}
	\item Supervised Learning
	\begin{itemize}
		\item Regression
		\begin{itemize}
			\item Simple Linear Regression
			\item Multiple Linear Regression
			\item Polynomial Regression
			\item Ridge Regression
			\item LASSO
			\item Kernel Regression (if time permits)
			\item Local Regression (if time permits)
		\end{itemize}
		\item Classification
		\begin{itemize}
			\item Nearest Neighbor Methods
			\item Naive Bayes
			\item Logistic Regression
			\item Decision Trees
			\item Random Forests
			\item Support Vector Machines
		\end{itemize}
	\end{itemize}
	\item Unsupervised Learning
	\begin{itemize}
		\item Dimensionality Reduction
		\begin{itemize}
			\item Principal Components Analysis
			\item t-SNE
		\end{itemize}
		\item Clustering
		\begin{itemize}
			\item k-Means
			\item Hierarchical
			\item DBScan
		\end{itemize}
	\end{itemize}
	\item Forecasting for Time Series Data
	\begin{itemize}
		\item Handling and cleaning time series data
		\item Simple forecasting methods
		\item Time series regression models
		\item Smoothing
		\item Exponential Smoothing
	\end{itemize}
	\item Neural Networks
	\begin{itemize}
		\item Perceptrons
		\item Shallow Networks
	\end{itemize}
	\item Presenting Results
	\begin{itemize}
		\item Pandas for presentation
		\item Advanced matplotlib
		\item Plotting in seaborn
		\item Introduction to Interactive Plotting With Python
	\end{itemize}
	\item Machine Learning Concepts
	\begin{itemize}
		\item Training Test Split
		\item Loss Functions
		\item Gradient Descent
		\item Model Validation
		\item Bias Variance Trade-Off
		\item Cross Validation
	\end{itemize}
\end{itemize}

\section*{Course Structure}

The boot camp will meet for twelve one and a half hour long sessions that are led
by an instructor. Most lectures will be accompanied by a homework assignment covering that
lecture's materials. At the end of the twelve lectures students will be given a week to
work on group projects. During that week lectures will be replaced with open office
hours in which lecturers and mentors may provide guidance for the group projects.
The boot camp will end with a presentation day for all group projects.

\subsection*{Lectures}
Each lecture the instructor will progress through a series of jupyter
notebooks as well as prepared python code. These sessions will feature a blend
of lecturing and working in small groups through coding examples. Students are
encouraged to ask questions during lectures in order to ensure they understand the
material.

\subsection*{Homework}
Most sessions will have a corresponding homework set that will highlight the material
covered in said session. Completion of homework sets is not mandatory, but highly encouraged. Working through problems and examples on your own will enrich your experience in the boot camp.

\subsection*{Group Projects}
Each participant must complete a group project by the end of the bootcamp. Projects
should feature the data science/machine learning skills taught in the bootcamp, or
even feature more advanced techniques groups teach themselves outside of bootcamp.

\section*{References}

All lectures and homeworks draw upon the material presented in the following books:
\begin{itemize}
	\item \href{http://shop.oreilly.com/product/0636920023784.do}{Python for Data Analysis}
	\item \href{https://www.oreilly.com/library/view/data-science-from/9781492041122/}{Data Science from Scratch}
	\item \href{https://www.oreilly.com/library/view/introduction-to-machine/9781449369880}{Introduction to Machine Learning with Python}
	\item \href{http://www.statlearning.com}{An Introduction to Statistical Learning}
	\item \href{https://link.springer.com/book/10.1007/978-0-387-84858-7}{The Elements of Statistical Learning}
	\item \href{"https://www.oreilly.com/library/view/hands-on-machine-learning/9781491962282/}{Hands-On Machine Learning with Scikit-Learn and TensorFlow}
	\item \href{https://www.oreilly.com/library/view/hands-on-unsupervised-learning/9781492035633/}{Hands-On Unsupervised Learning Using Python}
	\item \href{http://themlbook.com}{The Hundred-Page Machine Learning Book}
	\item \href{https://otexts.com/fpp2/}{Forecasting: Principles and Practice}
	\item \href{https://link.springer.com/chapter/10.1007/978-3-319-94463-0_1}{An Introduction to Neural Networks}
\end{itemize}
	
\end{document}